\chapter{Project}
\section{General}
\begin{description}
  \item[Euler equations:] discontinuous finite elements stabilized by entropy
    viscosity.
  \item[Radiative transfer:] discontinuous finite elements, matrix-free,
    need for fixup (?), preconditioned by (adapted?) MIP, \red{scalable sweeps
    on unstructured mesh}. 
  \item[JFNK and operator.]
  \item[((SD)I)RK methods:] use different time discretizations for different
    equations.
  \item[AMR:] one mesh per physics.
\end{description}
Usually implicit and explicit schemes are used together to decrease the
complexity of the computation. However, we need to be very careful if we want
to keep high order accuracy. 

In a Krylov linear solver the solution is built from a linear combination of
matrix vector products:
\begin{equation}
  \delta \bs{x}^k = \sum_{j=0}^{l-1} \alpha_j \bs{J}^j \bs{r}_0,
\end{equation}
where $\bs{r}_0$ is the initial residual to the linear problem, the
$\alpha_j$'s are the coefficients constructed by the Krylov methods, and $l$
is the number of Krylov iterations. \emph{Maybe just look in which cells the
residual is big and just compute the Jacobian for these cells. If the material
properties do not change to much, keep part of the previous Jacobian.}

\section{General structure}
\begin{description}
  \item[Material properties:] properties from table and/or equations.
    Different physics require different material properties.
    \begin{table}[H]
      \centering
      \begin{tabular}{|c|c|c|c|c|c|c|}
        \hline
        Interface & \multicolumn{6}{c|}{Material Properties} \\
        \hline
        Base class & \multicolumn{2}{c|}{Fluid properties} &
      \multicolumn{2}{c|}{R.T. properties} & \multicolumn{2}{c|}{Kinetic properties}
        \\
        \hline
        Derived class & Table & Formula & Table & Formula & Table & Formula \\
        \hline
      \end{tabular}
    \end{table}
  \item[Time discretization:] different discretizations coded by Butcher
    tables.
  \item[Operator Split:] Solve multiphysics problem using operator split.
  \item[Radiative transfer:] derived from Belos::Operator because it is matrix
    free.
  \item[Preconditioner:] base class for the preconditioners of the physics
    problems (ex. MIP).
  \item[Newton:] build the Jacobian or an approximation of the Jacobian if
    necessary and perform the Newton's iterations.
  \item[Preconditioner of the Jacobian:] preconditioner for the Jacobian.
  \item[Parameters:] contain all the parameters of the problems.
\end{description}

\section{Detailed structure}

\subsection{Material properties}
\begin{table}[H]
  \centering
  \begin{tabular}{|c|c|c|}
    \hline
    Public methods & Private methods & Members \\
    \hline
    Get\_fluid\_properties    & Read\_fluid\_data & *Fluid material properties \\
    Get\_radtrans\_properties & Read\_radtrans\_data & *Radiative transfer material properties \\
    Get\_kinetic\_properties  & Read\_kinetic\_data & *Kinetic material
    properties \\
    Read\_data & & \\
    \hline
  \end{tabular}
\end{table}

\subsection{Fluid material properties}
\begin{table}[H]
  \centering
  \begin{tabular}{|c|c|c|}
    \hline
    Public methods & Private methods & Members \\
    \hline
     & & \\
    \hline
  \end{tabular}
\end{table}

\subsection{Radiative transfer material properties}
\begin{table}[H]
  \centering
  \begin{tabular}{|c|c|c|}
    \hline
    Public methods & Private methods & Members \\
    \hline
    Get\_$\Sigma_t$ & Compute\_diffusion\_coefficient & $\Sigma_t$ \\
    Get\_$\Sigma_s$ & Compute\_$\Sigma_a$ & $\Sigma_s$ \\
    Get\_source     & Compute\_xs         & source \\   
    Set\_source     &                     & diffusion\_coefficient \\
    Get\_diffusion\_coefficient &         & $\Sigma_a$ \\
    \hline
  \end{tabular}
\end{table}

\subsection{Kinetic material properties}
\begin{table}[H]
  \centering
  \begin{tabular}{|c|c|c|}
    \hline
    Public methods & Private methods & Members \\
    \hline
     & & \\
    \hline
  \end{tabular}
\end{table}

\subsection{Time discretization}
\begin{table}[H]
  \centering
  \begin{tabular}{|c|c|c|}
    \hline
    Public methods & Private methods & Members \\
    \hline
    Get\_time\_step & Compute\_time\_step & name of ((SD(I))RK) \\
    Get\_weight     &                     & Butcher\_tableau    \\
    Get\_nodes      &                     & time\_ step         \\
    \hline
  \end{tabular}
\end{table}  

\subsection{Euler equations}
\begin{table}[H]
  \centering
  \begin{tabular}{|c|c|c|}
    \hline
    Public methods & Private methods & Members \\
    \hline
     & & \\
    \hline
  \end{tabular}
\end{table}

\subsection{Kinetic equations}
\begin{table}[H]
  \centering
  \begin{tabular}{|c|c|c|}
    \hline
    Public methods & Private methods & Members \\
    \hline
     & & \\
    \hline
  \end{tabular}
\end{table}

\subsection{Radiative transfer}
Derived from Belos::Operator. 
\begin{table}[H]
  \centering
  \begin{tabular}{|c|c|c|}
    \hline
    Public methods & Private methods & Members \\
    \hline
    Apply                       & Get\_saf   & communicator \\
    HasApplyTranspose           & Store\_saf & map \\
    Compute\_scattering\_source & Update\_saf & saf \\
    Sweep                       & & scattering\_src \\
                                & & radiative transfer material properties \\
                                & & quadrature \\
                                & & dof\_handler \\
                                & & triangulation \\
    \hline
  \end{tabular}
\end{table}

\subsection{Quadrature}
Base class for the quadratures used by the radiative transfer.
\begin{table}[H]
  \centering
  \begin{tabular}{|c|c|c|}
    \hline
    Public methods & Protected methods & Members \\
    \hline
    Build\_quadrature & Build\_octant           & galerkin \\
          Get\_n\_dir & Deploy\_octant          & sn \\
          Get\_n\_mom & Compute\_harmonics & L\_max \\
           Get\_omega &                         & n\_dir \\
             Get\_M2D &                         & weight \\
             Get\_D2M &                         & M2D \\
                      &                         & D2M \\
                      &                         & omega \\
    \hline
  \end{tabular}
\end{table}

\subsection{Level Symmetric quadrature}
Class derived from quadrature.
\begin{table}[H]
  \centering
  \begin{tabular}{|c|c|c|}
    \hline
    Public methods & Private methods & Members \\
    \hline
    & Build\_octant  & \\
    & Compute\_omega & \\
    \hline
  \end{tabular}
\end{table}

\subsection{Gauss-Legendre-Chebyshev quadrature}
Class derived from quadrature.
\begin{table}[H]
  \centering
  \begin{tabular}{|c|c|c|}
    \hline
    Public methods & Private methods & Members \\
    \hline
    & Build\_octant & \\
    & Build\_Chebyshev\_quadrature & \\
    \hline
  \end{tabular}
\end{table} 

\subsection{Operator split}
\begin{table}[H]
  \centering
  \begin{tabular}{|c|c|c|}
    \hline
    Public methods & Private methods & Members \\
    \hline
    Solve & Solve\_fluid    & flux\_moment \\
          & Solve\_radtrans & \\
          & Solve\_kinetic  & \\
    \hline
  \end{tabular}
\end{table}

\subsection{Mesh}
Create a namespace.
\begin{itemize}
  \item Mesh\_refinement
  \item Compute\_error (we don't want to use the Kelly Estimator for DG)
  \item Project\_on\_mesh (need to project on different mesh because of
    refinement/different physics)
\end{itemize}

\subsection{Output}
\begin{table}[H]
  \centering
  \begin{tabular}{|c|c|c|}
    \hline
    Public methods & Private methods & Members \\
    \hline
    Write\_output & & \\
    \hline
  \end{tabular}
\end{table}
