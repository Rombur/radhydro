\documentclass[letterpaper]{article}
\usepackage{amsmath}
\usepackage{array}
\usepackage{color}
\usepackage{graphicx}
\usepackage{float} % utiliser H pour forcer a mettre l'image ou on veut
\usepackage{lscape} % utilisation du mode paysage
\usepackage{mathbbol} % permet d'avoir le vrai symbol pour les reels grace a mathbb
\usepackage{enumerate} % permet d'utiliser enumerate
\usepackage{moreverb} % permet d'utiliser verbatimtab : conservation la tabulation
\usepackage{stmaryrd} % permet d'utiliser \llbrackedt et \rrbracket : double crochet
\usepackage[noabbrev]{cleveref} % permet d'utiliser cref and Cref
\usepackage{caption} % permet d'utiliser subcaption
\usepackage{subcaption} % permet d'utiliser subfigure, subtable, etc
\usepackage[margin=1.in]{geometry} % controle les marges du document


\newcommand\bn{\boldsymbol{\nabla}}
\newcommand\bo{\boldsymbol{\Omega}}
\newcommand\br{\mathbf{r}}
\newcommand\la{\left\langle}
\newcommand\ra{\right\rangle}
\newcommand\bs{\boldsymbol}
\newcommand\red{\textcolor{red}}
\newcommand\ldb{\{\!\!\{}
\newcommand\rdb{\}\!\!\}}
\newcommand\llb{\llbracket}
\newcommand\rrb{\rrbracket}

\renewcommand{\(}{\left(}
\renewcommand{\)}{\right)}
\renewcommand{\[}{\left[}
\renewcommand{\]}{\right]}


\begin{document}
\title{Radiation Hydrodynamic}
\author{Bruno Turcksin} 
\date{}
\maketitle

\section{Equations}
In this section, we will develop the equations for radiation hydrodynamics and
their discretization.
\subsection{Hydrodynamic}
First, we start with hydrodynamic equations.
\subsection{Continuous equations}
We want to solve the Euler equations:
\begin{align}
  &\frac{\partial \rho}{\partial t} + \bn (\rho \bs{u}) = 0 \label{mass},\\
  &\frac{\partial \rho \bs{u}}{\partial t} + \bn (\rho\bs{u}\bs{u}) + \bn\p =
  F \label{momentum},\\
  &\frac{\partial}{\partial t}\(\rho e +\frac{1}{2}\rho u^2\) +
  \bn\(\rho\bs{u}e + \frac{1}{2}\rho \bs{u} u^2+p\bs{u}\) = \rho q +\bs{u}F
  \label{energy},
\end{align}
where:
\begin{itemize}
  \item $\rho$ is the density,
  \item $t$ is the time,
  \item $\bs{u}$ is the speed,
  \item $\bs{u}\bs{u}$ is a tensor, the divergence is taken by assigning first
  1 to the second index, and forming the ordinary divergence of
  $\rho\bs{u}u_1$, then repeating for index 2 and index 3,
  \item $p$ is the pressure,
  \item $F$ is a body force,
  \item $e$ is the internal energy,
  \item $q$ is an external source of heat.
\end{itemize}
\Cref{mass} is the mass conservation equation, \cref{momentum} is the momentum
conservation equation, and \cref{energy} is the energy conservation equation.
The Euler equations describe an ideal fluid, i.e. with no viscosity and no heat
conduction. These equations allow shocks.
\subsubsection{Discretization}
Major discretizations:
\begin{enumerate}
  \item[Lagrangian:] the mesh follows the material. This is very efficient in
  one dimension but for multidimensional problems, the mesh can be too
  distorted and new it is necessary to remesh the domain.
  \item[Eulerian:] the mesh is fixed at the beginning of the calculation.
  Adaptive Mesh Refinement (AMR) is often necessary to have an accurate
  results.
  \item[Arbitrary Lagrangian-Eulerian (ALE):] the mesh moves at a given speed.
  \item[Smoothed particle hydrodynamics (SPH):] there is no mesh. SPH looks
  like a Monte-Carlo method but initial particles are not selected randomly.
  The method is first order.
\end{enumerate}

\section{Tests}

\section{State-of-the-art codes}

\end{document}
