\documentclass[letterpaper]{report}
\usepackage{amsmath}
\usepackage{array}
\usepackage{color}
\usepackage{graphicx}
\usepackage{float} % utiliser H pour forcer a mettre l'image ou on veut
\usepackage{lscape} % utilisation du mode paysage
\usepackage{mathbbol} % permet d'avoir le vrai symbol pour les reels grace a mathbb
\usepackage{enumerate} % permet d'utiliser enumerate
\usepackage{moreverb} % permet d'utiliser verbatimtab : conservation la tabulation
\usepackage{stmaryrd} % permet d'utiliser \llbrackedt et \rrbracket : double crochet
\usepackage[noabbrev]{cleveref} % permet d'utiliser cref and Cref
\usepackage{caption} % permet d'utiliser subcaption
\usepackage{subcaption} % permet d'utiliser subfigure, subtable, etc
\usepackage[margin=1.in]{geometry} % controle les marges du document


\newcommand\bn{\boldsymbol{\nabla}}
\newcommand\bo{\boldsymbol{\Omega}}
\newcommand\br{\mathbf{r}}
\newcommand\bu{\mathbf{u}}
\newcommand\la{\left\langle}
\newcommand\ra{\right\rangle}
\newcommand\bs{\boldsymbol}
\newcommand\red{\textcolor{red}}
\newcommand\ldb{\{\!\!\{}
\newcommand\rdb{\}\!\!\}}
\newcommand\llb{\llbracket}
\newcommand\rrb{\rrbracket}
\newcommand\mc{\mathcal}

\renewcommand{\(}{\left(}
\renewcommand{\)}{\right)}
\renewcommand{\[}{\left[}
\renewcommand{\]}{\right]}


\setcounter{secnumdepth}{3}

\begin{document}
\title{Radiation Hydrodynamic}
\author{Bruno Turcksin} 
\date{}
\maketitle

\chapter{Equations}
In this section, we will develop the equations for radiation hydrodynamics and
their discretization.
\section{Hydrodynamic}
First, we start with hydrodynamic equations.
\subsection{Continuous equations}
We want to solve the Euler equations:
\begin{align}
  &\frac{\partial \rho}{\partial t} + \bn (\rho \bs{u}) = 0 \label{mass},\\
  &\frac{\partial \rho \bs{u}}{\partial t} + \bn (\rho\bs{u}\otimes\bs{u}) + \bn p =
  F \label{momentum},\\
  &\frac{\partial}{\partial t}\(\rho e +\frac{1}{2}\rho u^2\) +
  \bn\(\rho\bs{u}e + \frac{1}{2}\rho \bs{u} u^2+p\bs{u}\) = \rho q +\bs{u}F
  \label{energy},
\end{align}
where:
\begin{itemize}
  \item $\rho$ is the density,
  \item $t$ is the time,
  \item $\bs{u}$ is the speed,
  \item $\bs{u}\otimes\bs{u}$ is a tensor, the divergence is taken by assigning first
    1 to the second index, and forming the ordinary divergence of
    $\rho\bs{u}u_1$, then repeating for index 2 and index 3,
  \item $p$ is the pressure,
  \item $F$ is a body force,
  \item $e$ is the internal energy,
  \item $q$ is an external source of heat.
\end{itemize}
\Cref{mass} is the mass conservation equation, \cref{momentum} is the momentum
conservation equation, and \cref{energy} is the energy conservation equation.
The Euler equations describe an ideal fluid, i.e. with no viscosity and no heat
conduction. These equations allow shocks.
\subsection{Discretization}
Major discretizations:
\begin{description}
  \item[Lagrangian:] the mesh follows the material. This is very efficient in
    one dimension but for multidimensional problems, the mesh can be too
    distorted and new it is necessary to remesh the domain.
  \item[Eulerian:] the mesh is fixed at the beginning of the calculation.
    Adaptive Mesh Refinement (AMR) is often necessary to have an accurate
    results.
  \item[Arbitrary Lagrangian-Eulerian (ALE):] the mesh moves at a given speed.
  \item[Smoothed particle hydrodynamics (SPH):] there is no mesh. SPH looks
    like a Monte-Carlo method but initial particles are not selected randomly.
    The method is first order.
\end{description}
In the rest of this paper, we will focus on Eulerian discretization.
\subsubsection{Lagrangian plus advection}
A time step consists of a ''Lagrangian'' step followed by an advection step.
The idea is the same than operator split: for each of these substeps just part
of the expression for the time derivative of each  is applied. The unknowns
$\rho$, $\rho\bs{u}$, and $\rho e$ are updated using those parts of their time
derivatives, and the updated values are the starting point for the next
partial time step. For the Lagrangian step the density is left alone, the
momentum density is updated using just the pressure gradient, after which a
new flow velocity is calculated by division. The velocity and momentum density
components are centered in time at $t_{n+1/2}$. The new velocities are used to
calculated $\bn\bs{u}$ and the internal energy density is updated using just
the first term on the right-hand side of \cref{3.22}. That completes the
Lagrangian step.

Equations for the Lagrangian step:
\begin{align}
  &\frac{\partial \rho}{\partial t} = -\bn (\rho \bs{u}),\\
  &\frac{\partial \rho \bs{u}}{\partial t} = -\bn p - \bn(\rho
  \bs{u}\otimes\bs{u}),\\
  &\frac{\partial \rho e}{\partial t} = -p\bn\bs{u} - \bn (\rho e \bs{u})
  \label{3.22}.
\end{align}
Equations for the advection step:
\begin{align}
  &\frac{d}{dt}(\rho V) = -\int_S \rho \bs{u}d\br,
  &\frac{d}{dt}(\rho \bs{u}V') = -\int_{S'} \rho \bs{u}\otimes\bs{u} d\br,
  &\frac{d}{dt}(\rho e V) = -\int_S \rho \bs{u}e d\br.
\end{align}
$V$ is a zonal volume and $V'$ is a dual zone volume, since the velocity
components are staggered in space from the quantities like density and
internal energy. The surface $S$ and $S'$ represent the boundaries of $V$ and
$V'$.
\subsubsection{Godunov's methods}
Godunov's methods are finite-volume methods. First, we write the Euler
equations as:
\begin{equation}
  \frac{\partial U}{\partial t} + \bn \bs{F}(U)=0.
\end{equation}
The vector $U$ contains the conserved densities $\rho$, $\rho u_x$, $\rho
u_y$, and $\rho e + \rho u^2/2$. The components of the flux vector are $\rho
\bs{u}$, $\rho \bs{u}u_x+ p\bs{e}_x$, $\rho \bs{u} u_y+p\bs{e}_y$, and $\rho
\bs{u} e +\rho \bs{u} u^2/2+p\bs{u}$. By averaging over a volume $V$ and over
a time interval $[t_n,t_{n+1}]$, we get:
\begin{equation}
  V\frac{\Delta \la U\ra}{\Delta t} = -\int_S\la \bs{F} \ra \cdot \bs{n} d\br.
\end{equation}
The expression $\la U \ra$ represents the zone average $U$ at the beginning
and the end of the time step; these are our unknowns. The quantity  $\la
\bs{F} \ra \cdot \bs{n}$ represents the flux of a conserved quantity through
the edge of a zone averaged over the time step. If we can construct accurate
values of these edge fluxes from the beginning-of-step values of $\la U \ra$,
then the advanced values follow immediately. From the components of $U$ it is
simple arithmetic to find $\bs{u}$ and $e$, and then the pressure can be
calculated. At the beginning of the time step we imagine that each zone is
uniform, with its average fluid quantities $\rho$, $\bs{u}$, and $p$. At an
interface between two zones there will thus be a discontinuity. The name for
the situation in which two regions of constant fluid properties meet at a
discontinuity is a Riemann problem. At the next instant this discontinuity
will be resolved into a pair of shocks, a shock and a rarefaction wave, or two
rarefaction waves, one traveling into each of the two zones. At the material
interface, which at the beginning of the time step coincides with the zone
boundary, there will be continuous values of pressure and velocity. Godunov's
method is to use these conditions, derived from the solution of the Riemann
problem, to evaluate the edge fluxes, them to use the conservation law to
update the conserved quantities to the next time step.
\subsubsection{WENO, TVD, TVB}
\begin{description}
  \item[WENO] stands for weighted essentially non-oscillatory. WENO consists
    of having different discretizations of the same cells and using a linear
    combination of them to have a non-oscillatory solution.
  \item[TVD] stands for total variation diminishing. It means the spatial
    fluctuation in the end of the time step cannot be larger than it was at the
    beginning of the time step.
  \item[TVB] stands for total variation bounded. It means that, while the
    fluctuation may grow, it will always be less than some fixed bound.
\end{description}

\section{Radiative transfer}
The equation is the ''usual'' equation:
\begin{equation}
  \frac{1}{c}\frac{\partial I}{\partial t} + \bo\cdot \bn I + \Sigma_t I =
  \int_{4\pi} \Sigma_s(\bo\cdot\bo') I\ d\bo' + Q,
\end{equation}
with $I$ is the intensity. Instead of using the energy as variable, we use the
frequency $\nu$.  We also define:
\begin{itemize}
  \item the energy density:
    \begin{equation}
      E = \frac{1}{c}\int_{4\pi} I\ d\bo.
    \end{equation}
  \item the flux:
    \begin{equation}
      \bs{F} = \int_{4\pi} \bo I\ d\bo.
    \end{equation}
  \item the pressure:
    \begin{equation}
      P = \frac{1}{c}\int_{4\pi}\bo\bo I\ d\bo.
    \end{equation}
\end{itemize}

\subsection{Comoving frame: transport equation}
We use the following definition:
\begin{equation}
  \Sigma_a = k_{\nu}
\end{equation}
Doppler and aberration relations:
\begin{align}
  \nu &= \nu_{(0)} \gamma_u \(1+\frac{\bo_{(0)}\cdot \bs{u}}{c}\),\\
  \bo &= \frac{\gamma_u\frac{\bs{u}}{c}+\bo_{(0)}+(\gamma_u-1)(\bo_{(0)}\cdot
  \bs{u})\bs{u}/u^2}{\gamma_u(1+\bo_{(0)}\cdot \bs{u}/c)},
\end{align}
where $\gamma_u = \frac{1}{\sqrt{1-\frac{u^2}{c^2}}}$. A fixed $\bs{r}_0$ is 
moving with the velocity $\bs{u}$ in the $(t,\bs{r})$ coordinates. For small 
values of the relative velocity $\bs{u}$ of the two frames these relations
simplify considerably. In that case we can neglect all the $u^2/c^2$ terms,
which means that $\gamma_u$ can be replaced by 1. Therefore:
\begin{align}
  \nu &= \nu_{(0)}\(1+\frac{\bo_{(0)}\cdot \bs{u}}{c}\),\\
  \bo &= \frac{\bo_{(0)} + \bs{u}/c}{1+\bo_{(0)}\cdot \bs{u}/c}.
\end{align}
Discarding all the terms that are $O(u^2/c^2)$ or higher. This result is
obtained:
\begin{equation}
  \begin{split}
    &\(1+\frac{\bo_{(0)}}{c}\cdot\bs{u}\)\(\frac{1}{c} \frac{\partial
    I^{(0)}}{\partial t}+\frac{\bs{u}}{c}\cdot\bn I^{(0)}\) + \bo_{(0)} \cdot \bn
    I^{(0)}\\
    &-\frac{\nu_{(0)}}{c}\(\frac{\bs{a}}{c}+\bo_{(0)}\cdot\bn\bs{u}\) \cdot
    \bn_{\nu_{(0)}\bo_{(0)}}I^{(0)}\\
    &+\frac{3}{c}\(\frac{\bo_{(0)}\cdot\bs{a}}{c} + \bo_{(0)}\cdot \bn
    \bs{u}\cdot \bo_{(0)}\) I^{(0)} = j^{(0)} - k^{(0)}I^{(0)}.
  \end{split}
  \label{6.39}
\end{equation}
For simplicity here and below, the subscripts indicating that all the
frequency-dependent quantities in the comoving frame are evaluated at
$\nu_{(0)}$ have been omitted. There are two features of this equation in
which we should comment first. The term $\bo_{(0)}\cdot \bs{u}/c$ in the
coefficient of the first term would go away if the transformation from the
fixed frame to the comoving frame had really been a Lorentz transformation,
since a spatial derivative at constant time in the fixed frame owing to the
relativity of time. But we are not altering the time coordinate in our
procedure, so this term is left over. The terms involving the acceleration,
$\bs{a}$, arise from $\frac{\partial \bs{u}}{\partial t}$. The actual
acceleration is $\frac{\partial \bs{u}}{\partial t} + \bs{u}\cdot\bn\bs{u}$,
of course, so we might wonder about the other piece. But the extra parts would
bring in terms that are of order $u^2/c^2$, and all such terms have been
discarded. It is our opinion that the acceleration terms, as well as the
$\bo_{(0)}\cdot \frac{\bs{u}}{c}$ part of the coefficient of the first term
should be discarded. The reasoning is as follows. In a fluid flow problem the
time derivative and the flow derivative $\bs{u}\cdot\bn$ are generally of the
same order of magnitude. If the acceleration terms above are ordered in this
way then all of them are seen to be of order $u^2/c^2$ compared with the
dominant terms, and may be discarded. The same is true of the
$\bo_{(0)}\cdot\frac{\bs{u}}{c}$ term multiplying $\frac{\partial
I^{(0)}}{\partial t}$.

We suggest that \cref{6.39} be retained for those problems involving
nonrelativistic velocities that evolve on a light-transit scale, and we
simplify this equation by dropping the subject the subject terms for problems
with the fluid-flow time scale. Carrying out the simplification leads to:
\begin{equation}
  \begin{split}
    &\frac{1}{c}\frac{D I^{(0)}}{Dt} +\bo_{(0)}\cdot\bn
    I^{(0)}-\frac{\nu_{(0)}}{c}\bo_{(0)}\cdot\bn\bs{u}\cdot
    \bn_{\nu_{(0)}\bo_{(0)}}I^{(0)}\\
    &+\frac{3}{c}\bo_{(0)}\cdot\bn\bs{u}\cdot\bo_{(0)}I^{(0)} =
    j^{(0)}-k^{(0)}I^{(0)}
  \end{split}
\end{equation}
where $\frac{D}{Dt}=\frac{\partial}{\partial t}+\bs{u}\cdot\bn$.

Common-sense summary:
\begin{itemize}
  \item Ignoring the Doppler effect entirely and using the fixed-frame
    transport equation with the absorptivity and emissivity appropriate for
    matter at rest gives the wrong answer for the radiation when the velocity is
    supersonic and spectral  lines dominate the opacity, and it also makes a
    significant error in the energy coupling rate of radiation and matter.
  \item \Cref{6.35,6.36} have to be used to get the correct coupling terms for
    the fixed-frame equations.
  \item The coupling term in the material internal energy equation is indeed
    the fluid-frame energy term $g_{(0)}^0$ from \cref{6.37}, but in the
    material total energy equation it is combined in with the work done by the
    radiation force, which turns it into the fixed-frame energy term equation.
  \item The diffusion limit gives a Fick's law form $(\bs{F}\propto -\bn E)$
    for the flux in the fluid frame, not the fixed frame. In the fixed frame
    the convective flux of radiation enthalpy is added on.
  \item Confusion of the correct frames for the coupling terms and the Fick's
    law flux is more serious than the other moderate corrections the $u/c$ terms
    give.
  \item The velocity terms in the comoving-frame energy equation matter,
    especially the Doppler-shift frequency-derivative term; the ones in the
    comoving-frame momentum equation are less important.
  \item The velocity terms in the comoving-frame energy equation matter,
    especially the Doppler-shift frequency-derivative term; the ones in the
    comoving-frame momentum equation are less important. The aberration
    (angle-derivative) term in the transport equation does not survive in the
    energy moment and dos not matter in the flux moment, so this can be dropped
    with little consequence.
\end{itemize}

\subsection{Comoving-frame: diffusion equation}
We have mentioned several times now that the diffusion approximation should be
applied in the comoving frame if the intent is to obtain a flux that tends to
zero as the mean free path becomes small, i.e., one which obeys Fick's ;aw. We
will substantiate that claim by rederiving the diffusion results beginning
with the comoving-frame transport equation. We begin with:
\begin{equation}
  \begin{split}
    I^{(0)} =& \frac{j^{(0)}}{k^{(0)}}-\frac{1}{k^{(0)}}\(\(1+
    \frac{\bo_{(0)}}{c}\cdot\bu\)\(\frac{1}{c}\frac{\partial I^{(0)}}{\partial
    t} +\frac{\bu}{c}\cdot\bn I^{(0)}\) + \bo_{(0)}\cdot \bn I^{(0)}\right.\\
    &\left.-\frac{\nu_{(0)}}{c}\(\frac{\bs{a}}{c} + \bo_{(0)}\cdot \bn \bu\)\cdot
    \bn_{\nu_{(0)}\bo_{(0)}} I^{(0)} +\frac{3}{c} \(\frac{\bo_{(0)}\cdot
    \bs{a}}{c} + \bo_{(0)} \cdot \bn\bu\cdot\bo_{(0)}\)I^{(0)}\).
    \label{6.57}
  \end{split} 
\end{equation}
The zeroth approximation is that $I^{(0)}$ is $\frac{j^{(0)}}{k^{(0)}}$ which
we identify with the thermodynamic equilibrium value, the Planck function:
\begin{equation}
  \frac{j^{(0)}}{k^{(0)}} = B_{\nu}(T).
  \label{6.58}
\end{equation}
Making this replacement for $I^{(0)}$ in \cref{6.57} leads to:
\begin{equation}
  \begin{split}
    I^{(0)} =& B_{\nu}(T)-\frac{1}{k^{(0)}} \(\(1+\frac{\bo_{(0)}}{c}\cdot
    \bu\)\(\frac{1}{c}\frac{\partial B_{\nu}}{\partial t} +
    \frac{\bu}{c}\cdot\bn B_{\nu}\) + \bo_{(0)}\cdot \bn B_{\nu}\right.\\
    &\left.-\frac{\nu_0}{c}\(\frac{\bs{a}}{c}+\bo_{(0)}\cdot\bn\bu\)\cdot
    \bn_{\nu_{(0)}\bo_{(0)}}B_{\nu} +
    \frac{3}{c}\(\frac{\bo_{(0)}\cdot\bs{a}}{c} +\bo_{(0)}\cdot\bn \bu
    \cdot\bo_{(0)}\) B_{\nu}\). 
    \label{6.59}
  \end{split}
\end{equation}
Now the task is to expand \cref{6.59} and keep terms of first order in the
velocity. The Planck function is isotropic, but it does have spatial and
temporal gradients. Its momentum-space gradient is:
\begin{equation}
  \bn_{\nu_{(0)}\bo_{(0)}} B_{\nu} = \frac{\partial
  B_{\nu}}{\partial\nu}\bo_{(0)}.
  \label{6.60}
\end{equation}
What results in:
\begin{equation}
  \begin{split}
    I^{(0)} =&B_{\nu}-\frac{1}{k^{(0)}} \(\frac{dB_{\nu}}{dT}
    \(\frac{1}{c}\frac{DT}{Dt}+\bo_{(0)}\cdot \bn T +
    \frac{\bo_{(0)}\cdot\bu}{c^2}\frac{DT}{Dt}\)\right.\\
    &\left.-\frac{1}{c}\(\frac{\bs{a}\cdot\bo_{(0)}}{c}+\bo_{(0)}\cdot\bn\bu
    \cdot \bo_{(0)}\) \nu_{(0)}\frac{\partial B_{\nu}}{\partial \nu} +
    \frac{3}{c}\(\frac{\bo_{(0)}\cdot\bs{a}}{c} + \bo_{(0)}\cdot\bn\bu\cdot
    \bo_{(0)}\)B_{\nu}\).
    \label{6.61}
  \end{split}
\end{equation}
Integrating \cref{6.61} over angles leads to the diffusion formula for the
comoving-frame monochromatic energy density:
\begin{equation}
  E_{\nu}^{(0)} = \frac{4\pi B_{\nu}}{c} - \frac{4\pi}{k_{\nu}^{(0)} c^2}
  \(\frac{dB_{\nu}}{dT}\frac{DT}{Dt}+ \frac{\bn\cdot \bu}{3}
  \(3B_{\nu}-\nu\frac{\partial B_{\nu}}{\partial nu}\)\).
  \label{6.62}
\end{equation}
By virtue of Wien's displacement law, which is:
\begin{equation}
  B_{\nu}(T) = \nu^3 \times\textrm{function}(\nu/T),
  \label{6.63}
\end{equation}
It follows that:
\begin{equation}
  3 b_{\nu} -\nu\frac{\partial B_{\nu}}{\partial \nu}=T\frac{dB_{\nu}}{dT}.
  \label{6.64}
\end{equation}
Making this replacement in \cref{6.62} leads to the simple form:
\begin{equation}
  E_{\nu}^{(0)} = \frac{4\pi B_{\nu}}{c} -
  \frac{4\pi}{k_{\nu}^{(0)}c^2}\frac{dB_{\nu}}{dT}
  \(\frac{DT}{Dt}+\frac{\bn\cdot\bu}{3} T\).
  \label{6.65}
\end{equation}
A frequency integration of \cref{6.65} leads to the following expression for
the total comoving-frame energy density:
\begin{equation}
  E = \frac{4\pi B}{c} \(1-\frac{4}{k_R c}\(\frac{1}{T}\frac{DT}{Dt} +
  \frac{\bn\cdot\bu}{3}\)\),
  \label{6.66}
\end{equation}
where the Rosseland mean has been put in to replace its defining integral:
\begin{equation}
  \frac{1}{k_R} = \frac{\int d\nu\ \(\frac{1}{k_{\nu}^{(0)}}\)
  \frac{dB_{\nu}}{dT}}{\int d\nu\ \frac{dB_{\nu}}{dT}},
  \label{6.67}
\end{equation}
and where the notation $B=\int d\nu\ B_{\nu}$ has been used. In fact, $4\pi
\frac{B}{c}$ is just the thermodynamic-equilibrium radiation energy density
$aT^4$, and the fact that $\frac{dB}{dT} = 4\frac{B}{T}$ has been used in
deriving \cref{6.66}. If the characteristic length scale for the problem is
$L$ and the characteristic time is $\tau$, then the correction term in
\cref{6.66} is of order $\frac{\lambda_R}{c\min(\tau,L/u)}$, where $\lambda_R$
denotes the Rosseland mean of the mean free path. So if the flow $L/u$ is
longer than the characteristic time $\tau$ then the order of the correction is
$\frac{\lambda_R}{c\tau}$, while if the flow time is shorter then the order is
$(\lambda_R/L)(u/c)$. If the characteristic time is so short that it is
comparable with the light-transit time $c/L$, then the order is just
$\lambda_R/L$. When the flow time is shortest the size of the correction is
the product of two small quantities, $\lambda_R/L$ and $u/c$. The
monochromatic flux derived from \cref{6.61} is:
\begin{equation}
  \bs{F}_{\nu}^{(0)}=-\frac{4\pi}{3k_{\nu}^{(0)}} \frac{dB_{\nu}}{dT} \(\bn T
  +\frac{\bu}{c^2} \frac{DT}{Dt} +\frac{\bs{a}}{c^2} \frac{DT}{Dt} +
  \frac{\bs{a}}{c^2} T\),
  \label{6.68}
\end{equation}
where we have again been able to use Wien's displacement law \cref{6.64}. We
have the problem now of finding the physical explanations for the
at-first-sight puzzling terms in the velocity and the acceleration. The
velocity is relatively easy. We need to need to recall that we are using the
fixed-frame coordinates, and that in the Lorentz transformation to a locally
comoving frame not only does the time derivative transform according to:
\begin{equation}
  \frac{\partial}{\partial t}\rightarrow \frac{D}{Dt} =
  \frac{\partial}{\partial t}+\bu\cdot\bn ,
  \label{6.69}
\end{equation}
but the spatial derivative changes from one at constant $t$ to one at constant
comoving time $t'$ according to the other Lorentz relation:
\begin{equation}
  \bn\rightarrow \bn'=\bn+\frac{\bu}{c^2}\frac{\partial}{\partial t}.
  \label{6.70}
\end{equation}
As we see, the $\bu$ term in \cref{6.68} is absorbed in changing $\bn T$ to
$\bn'T$, apart from an error that is $O(u^2/c^2)$. In short, this velocity
term is real and represents a relativistic effect on the diffusion flux. The
acceleration term has a physical interpretation as well. The mean free path
$1/k_{\nu}^{(0)}$ corresponds to a flight time $1/\(k_{\nu}^{(0)}c\)$ during
which the local fluid velocity has increased by an amount
$\bs{a}/\(k_{\nu}^{(0)}c\)$. The incremental bosst by this velocity change
makes a contribution to the flux of $-\bs{a}/\(k_{\nu}^{(0)}\)$ times the sum
of $E$ and $P$, which gives the term in question. To put it another way, if
the flux is zero at the time when the photon flight begins, then by the end of
the flight the fluid as accelerated away from the rest of the radiation, which
produces a flux in the new fluid rest frame in the direction opposite to the
acceleration that is proportional to the acceleration times the flight time.
The frequency-integrated flux is simply:
\begin{equation}
  \bs{F}^{(0)} = -\frac{4\pi}{3k_R}\frac{dB}{dT}\(\bn'T+\frac{\bs{a}}{c^2}T\),
  \label{6.71}
\end{equation}
which, since $B=acT^4/(4\pi)$, can be written:
\begin{equation}
  \bs{F}^{(0)} = -K_R\(\bn'T+\frac{\bs{a}}{c^2}T\),
  \label{6.72}
\end{equation}
in terms of the Rosseland-mean radiative conductivity:
\begin{equation}
  K_R = \frac{4\pi}{3k_R}\frac{dB}{dT}=\frac{4acT^3}{3k_R}.
  \label{6.73}
\end{equation}
The gradient operator here is the Lorentz-corrected one from \cref{6.70}.
Before writing the result for the pressure tensor we have to obtain the fully
symmetric angle average of the product of four $\bo$s. In other words, we want
to evaluate:
\begin{equation}
  \frac{1}{4\pi}\int_{4\pi}d\bo \Omega_i\Omega_j\Omega_k\Omega_l.
  \label{6.74}
\end{equation}
First we note that at least two of the four indices must be equal, since there
are four indices and only three possible values they can take. If one pair out
of the four are equal, say $i=j$, then it must be true that the other pair are
equal as well, $k=l$, or otherwise the integral vanishes. Thus one
contribution to the integral is proportional to $\delta_{ij}\delta_{kl}$. By
choosing the first pair in the three possible ways and adding the results we
get the required fully-symmetric form:
\begin{equation}
  \delta_{ij}\delta_{kl}+\delta_{ik}\delta_{jl}+\delta_{il}\delta_{jk}.
  \label{6.75}
\end{equation}
The integral must be proportional to this. We get the proportionality factor
by taking the case $i=j=k=l=1$, for which the angle average of $\Omega_x^4$ is
seen to be $1/5$ by doing the integrations in that case, and for which the sum
of delta functions is 3. Thus:
\begin{equation}
  \frac{1}{4\pi}\int_{4\pi}d\bo\Omega_i\Omega_j\Omega_k\Omega_l =
  \frac{1}{15}\(\delta_{ij}\delta_{kl}+\delta_{ik}\delta_{jl}+
  \delta_{il}\delta_{jk}\).
  \label{6.76}
\end{equation}
Now we can do the integrations to fin the monochromatic pressure tensor:
\begin{equation}
  \begin{split}
    P_{\nu}^{(0)} =&\(\frac{4\pi B_{\nu}}{c}-\frac{4\pi}{k_{\nu}^{(0)}c^2}
    \frac{dB_{\nu}}{dT}\frac{DT}{Dt}\)\frac{\mc{I}}{3}\\
    -\frac{4\pi}{15k_{\nu}^{(0)}c^2}T\frac{dB_{\nu}}{dT}\(\bn\bu+\(\bn\bu\)^T+
    \bn\cdot\bu\mc{I}\),
  \end{split}
  \label{6.77}
\end{equation}
and the frequency-integrated form:
\begin{equation}
  P^{(0)} = \frac{1}{3}aT^4\(1-\frac{4}{k_Rc}\frac{1}{T}\frac{DT}{Dt}\)\mc{I}
  -\frac{4aT^4}{15k_Rc}\(\bn\bu+\(\bn\bu\)^T+\bn\cdot\bu\mc{I}\).
  \label{6.78}
\end{equation}
The form of this pressure tensor implies that radiation in the diffusion limit
contributes normal and bulk viscosity coefficients:
\begin{equation}
  \mu_R = \frac{4aT^4}{15k_R c}
  \label{6.79}
\end{equation}
and:
\begin{equation}
  \xi_R =\frac{5}{3}\mu_R =\frac{4aT^4}{9k_Rc}
  \label{6.80}
\end{equation}
to the mixed fluid. The bulk viscosity in this picture is not at all close to
zero, which is a cause of consternation in view of some very general
relativistic results.

The order of magnitude of the shear stress contribution to $P$ compared with
the isotropic pressure is $O(\lambda_R u/(Lc))$ in the notation used above. In
other words, isotropy of $P$ is a very good approximation in a diffusion
regime owing to the smallness of both $\lambda_R/L$ and $u/c$.

\section{Radiative hydrodynamic}
\subsection{Coupling terms in Euler's equation}
First, we need to define:
\begin{align}
  &g^0 = \int d\nu\int_{4\pi} d\bo (S-\Sigma_a I),\\
  &\bs{g} = \frac{1}{c}\int d\nu \int_{4\pi} d\bo \bo(S-\Sigma_a I).
\end{align}.
We should put the negatives of these on the right-hand side of the material
energy and momentum equations:
\begin{align}
  &\frac{\partial}{\partial t}\(\rho e + \frac{1}{2}\rho u^2\) + \bn \(\rho
  \bs{u} e + \frac{1}{2} \rho \bs{u}u^2\) = - g^0,\\
  & \frac{\partial \rho \bs{u}}{\partial t} + \bn(\rho\bs{u}\otimes\bs{u})+
  \bn p = 0\bs{g}.
\end{align}
We have left out the nonradiative body force and heat deposition terms for
simplicity. Summing the frequency-integrated radiation moment equations and
the material equations gives the overall conservation laws:
\begin{align}
  &\frac{\partial}{\partial t}\(\rho e + E + \frac{1}{2}\rho u^2\) + \bn \(\rho
  \bs{u} e + \frac{1}{2} \rho \bs{u}u^2 + \bs{F}\) = 0,\\
  & \frac{\partial}{\partial t}\(\rho \bs{u} + \frac{\bs{F}}{c^2}\) +
  \bn(\rho\bs{u}\otimes\bs{u}+ P) + \bn p = 0.
\end{align}

\subsection{Comoving frame}
$g^0$ and $\bs{g}$ are exactly as given earlier. The Lorentz transformation to
first order in $u/c$ is:
\begin{align}
  &g^0 = g_{(0)}^0 + \frac{\bs{u}}{c^2}\cdot \bs{g}_{(0)}, \label{6.35}\\
  &\bs{g} = \bs{g}_{(0)} + \bs{u} g_{(0)}^0. \label{6.36}
\end{align}
The second term on the right-hand side of the $\bs{g}$ equation is another of
those pesky ones that does not have an analog in nonrelativistic mechanics. It
comes about because the addition of energy increase the relativistic mass
density and therefore also the relativistic momentum density. When we treat
the material fluid nonrelativistically there is no similar term in the
material density. When we treat the material fluid nonrelativistically there
is no similar term in the material momentum equation to compensate this one.
Fortunately, it is small and we can discard it or hide it somewhere. The
velocity term in the $g^0$ equation does have a nonrelativistic meaning. It is
the rate of doing work by the force exerted on the radiation by the matter.
The components of $g^{\mu}$ in the fluid frame are easily found from the
atomic properties of the matter, without the need for Doppler and aberration
transformations. Furthermore, we can almost always assume isotropy of the
absorptivity and emissivity in the fluid frame. With that assumption, we
obtain:
\begin{align}
  &g_{(0)}^0 = \int d\nu\ \(4\pi
  j_{\nu}^{(0)}-\Sigma_{a,\nu}^{(0)}cE_{\nu}^{(0)}\), \label{6.37}\\
  &\bs{g}_{(0)} = -c \int d\nu\ \Sigma_{a,\nu}^{(0)}\bs{F}_{\nu}^{(0)}.
  \label{6.38} 
\end{align}
These are the same expressions that we used for the right-hand sides of the
fixed frame movement equations earlier, only repeated here using the
fluid-frame absorptivity and emissivity and the fluid-frame radiation moments.
The important difference is that the earlier fixed-frame expressions are wrong
if there is an appreciable velocity while the fluid-frame expressions are
correct as long as the velocity is non-relativistic. The components $g^0$ and
$\bs{g}$ obtained by substituting \cref{6.37,6.38} into \cref{6.35,6.36} give
the correct quantities to put in the right-hand sides of the fixed-frame
moments equations.

\subsection{Local Thermodynamic Equilibrium (LTE)}
Assumptions when using Local Thermodynamic Equilibrium:
\begin{itemize}
  \item inhomogeneous temperature but small gradient of temperature.
  \item thermal equilibrium for the matter only not for the radiation.
\end{itemize}
Resonance line scattering is a kind of inelastic scattering: a photon is
absorbed and then, remitted. Effects important for LTE:
\begin{description}
  \item[Spontaneous emission:] Einstein coefficients $A_{21}$. $A_{21}$
    is the transition rate for spontaneous emission.
  \item[Absorption:] Einstein coefficients $B_{12}$. $B_{12}J_{\nu_0}$ is the
    transition rate for absorption.
  \item[Stimulated emission:] Einstein coefficient $B_{21}$. $B_{21}J_{\nu_0}$
    is the transition rate for stimulated emission.
\end{description}
As a system for doing calculations, non-LTE consists of solving for the
radiation field $I_{\nu}$ and the set of atomic occupations $N_{ij}$, also
often called the ''populations''. Here $i$ is an index for an energy level
belonging to that element. The set of equations that determines these are the
transport equation, which fixes the intensity, and the atomic kinetic
equations. The kinetic equations have this generic form:
\begin{equation}
  \rho\frac{D\(N_{ij}/\rho\)}{Dt} = \sum_k\(N_{ik}\(P_{kj}+C_{kj}\) - N_{ij}
  \(P_{jk}+C_{jk}\)\).
  \label{9.1}
\end{equation}
The $P$s and $C$s are rate coefficients. The $P$s in particular are radiative
rates, such as spontaneous and stimulated decay, photoabsorption, and
photoionization, and the $C$s are collisional rates, most commonly the ones for
inelastic electron collisions. The elastic rates are ignored, since they do
not change the level populations (although they count in electron impact line
broadening). The electron rates can be expressed as $N_e$ times a collisional
rate coefficient, and the latter is a quantity like $\la\nu\sigma_{jk}\ra$,
where the brackets indicate an average over the electron Maxwellian
distribution of velocities. The radiative rates are combinations of
spontaneous downward rates, stimulated downward rates that are written as
integrals of appropriate cross sections over the photon flux spectrum, and
upward radiative rates that are likewise integrals of photon fluxes time cross
sections.

Every rate in \cref{9.1} appears twice, once a positive term in the equation
for the receiving level, and a second time as a negative term in the equation
for the originating level. As a result, if all the equations for the levels
of alls the ions of one element are added up, everything in the sum cancels
out. The time derivative on the left turns into $\rho$ times the convective
derivative of the atomic fraction of that element, which vanishes unless there
is diffusion (of atoms) in the fluid.

\section{Multiphysic problem}
\subsection{Operator split}
We want to solve the coupled system:
\begin{align}
  &A(y) x = b, \label{multiphysic_1}\\
  &B(x) y = c. \label{multiphysic_2}
\end{align}
First, we need to fix the value of $y$ to solve:
\begin{equation}
  A(y^*) x = b.
  \label{os_1}
\end{equation}
The solution of \cref{os_1}, $x_{sol}$ is used to solve:
\begin{equation}
  B(x_{sol}) y = c.
\end{equation}
\subsection{Newton's method}
First, let's rewrite \cref{multiphysic_1,multiphysic_2}, as:
\begin{equation}
  F(z) = 0.
  \label{multiphysic_3}
\end{equation}
\Cref{multiphysic_3} is a nonlinear system that can solve using Newton's
method:
\begin{equation}
  J(z_n) (z_{n+1}-z_n) = -F(z_n),
\end{equation}
where $J$ is the Jacobian of the problem and $F(z_n)$ is the residual.

\section{Other Physics}
Other phenomena that you be taken into account: gravity, polarization, and
magnetic field.



\chapter{Tests}
\section{Marshak wave}
\section{I-front}
\section{Radiating shock waves}



\chapter{Project}
\section{General}
\begin{description}
  \item[Euler equations:] discontinuous finite elements stabilized by entropy
    viscosity.
  \item[Radiative transfer:] discontinuous finite elements, matrix-free,
    need for fixup (?), preconditioned by (adapted?) MIP.
  \item[JFNK and operator.]
  \item[((SD)I)RK methods:] use different time discretizations for different
    equations.
  \item[AMR:] one mesh per physics.
\end{description}

\section{General structure}
\begin{description}
  \item[Material properties:] properties from table and/or equations.
    Different physics require different material properties.
    \begin{table}[H]
      \centering
      \begin{tabular}{|c|c|c|c|c|c|c|}
        \hline
        Interface & \multicolumn{6}{c|}{Material Properties} \\
        \hline
        Base class & \multicolumn{2}{c|}{Fluid properties} &
      \multicolumn{2}{c|}{R.T. properties} & \multicolumn{2}{c|}{Kinetic properties}
        \\
        \hline
        Derived class & Table & Formula & Table & Formula & Table & Formula \\
        \hline
      \end{tabular}
    \end{table}
  \item[Time discretization:] different discretizations coded by Butcher
    tables.
  \item[Physics:] base class for the physics: wrapper around
    Belos::LinearProblem so that it is easy to have a matrix-free physics
    $\rightarrow$ implement Tpetra and Belos in deal.II. Used for operator
    split. The different physics are derived from this class.
  \item[Radiative transfer:] derived from physics.
  \item[Preconditioner:] base class for the preconditioners of the physics
    problems (ex. MIP).
  \item[Newton:] build the Jacobian or an approximation of the Jacobian if
    necessary and perform the Newton's iterations.
  \item[Preconditioner of the Jacobian:] preconditioner for the Jacobian.
  \item[Parameters:] contain all the parameters of the problems.
\end{description}

\section{Detailed structure}
\subsection{Radiative transfer material properties}
\begin{table}[H]
  \centering
  \begin{tabular}{|c|c|c|}
    \hline
    Public methods & Private methods & Members \\
    \hline
                   &                 & $\Sigma_t$ \\
                   &                 & $\Sigma_s$ \\
    \hline
  \end{tabular}
\end{table}



\end{document}
